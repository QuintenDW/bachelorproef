%%=============================================================================
%% Inleiding
%%=============================================================================

\chapter{\IfLanguageName{dutch}{Inleiding}{Introduction}}%
\label{ch:inleiding}


In de loop der jaren zijn applicaties steeds complexer geworden. Hierbij worden achterliggend netwerkverzoeken afgehandeld door een backend applicatie. Deze applicatie draait in een runtime-omgeving buiten de browser, 
waardoor deze toegang heeft tot databanken en netwerken van de server waarop het wordt uitgevoerd.
Een manier om zo een backend te ontwikkelen is met behulp van JavaScript, 
een programmeertaal die zowel voor server als client toepassingen kan gebruikt worden.
Backend JavaScript applicaties worden dan uitgevoerd in een JavaScript runtime-omgeving.
De meest populaire omgeving is Node.js. Zo toont onderzoek van ~\textcite{Greif2022} aan dat 93.6\% van de 29888 bevraagden Node.js 
regulier gebruiken.
Dit tegenover de 11.2\% en 4.3\% van de mensen die andere omgevingen zoals respectievelijk Deno en Bun gebruiken.
Deze 2 omgevingen zijn relatief jonger dan Node.js.

\section{\IfLanguageName{dutch}{Probleemstelling}{Problem Statement}}%
\label{sec:probleemstelling}

Performantie is doorheen de tijd alsmaar belangrijker geworden. 
Een performante applicatie zorgt hierbij voor een betere gebruikerservaring. 
Momenteel blijft Node.js de standaard keuze voor veel projecten. Niettemin zijn er nog tal van andere JavaScript runtime-omgevingen
, zoals Bun en Deno, die in staat zijn om de specifieke noden te vervullen waar Node.js niet aan kan voldoen. 
In dit onderzoek wordt dieper ingegaan op de performantie van Node.js en één van deze nieuwe omgevingen,
om zo een weloverwogen keuze te kunnen maken bij de ontwikkeling van performante applicaties.
Dit onderzoek is relevant voor bedrijven zoals Codifly, waar performantie een essentiële rol speelt binnen de ontwikkeling van JavaScript backend applicaties.
De bevindingen kunnen dienen als basis 
voor het maken van een doordachte backend keuze. Dit onderzoek 
biedt ook inzicht in de capaciteiten van Bun voor andere betrokkenen binnen het vakgebied.

\section{\IfLanguageName{dutch}{Onderzoeksvraag}{Research question}}%
\label{sec:onderzoeksvraag}

In dit onderzoek wordt bestudeerd of één van deze nieuwe omgevingen
een correcte plaatsvervanger kan zijn voor Node.js bij de ontwikkeling van performante applicaties binnen bedrijven 
waar een backend JavaScript runtime wordt gebruikt.
Hierbij spelen verschillende aspecten een rol die aan de hand van volgende deelvragen worden onderzocht:
\begin{itemize}
  \item Wat zijn de onderliggende verschillen tussen Node.js en de andere omgeving?
  \item Is de gekozen omgeving performanter dan Node.js bij het uitvoeren van computationele berekeningen?
  \item Is de gekozen omgeving performanter dan Node.js bij het afhandelen van netwerkverzoeken?
  \item Wat is het verschil tussen hun respectievelijke package managers?
\end{itemize}

\section{\IfLanguageName{dutch}{Onderzoeksdoelstelling}{Research objective}}%
\label{sec:onderzoeksdoelstelling}

Dit onderzoek heeft als doel inzicht te verschaffen in de performantie van Node.js en één van deze nieuwe omgevingen,
zodat een correcte keuze kan worden gemaakt bij het ontwikkelen van performante applicaties. Dit wordt bereikt door middel van een vergelijkende studie
waarbij zowel de geselecteerde omgeving als Node.js worden getest aan de hand van proof-of-concepts. 
Hierbij worden de prestaties gemeten tijdens computationele berekeningen en het afhandelen van netwerkverzoeken, waarbij specifiek gekeken wordt naar
de latentie, aantal verzoeken per seconde, installatietijd en uitvoeringstijd. 
Het doel is deze resultaten te gebruiken om vast te stellen of de gekozen omgeving performanter is dan Node.js. 
Het onderzoek wordt als succesvol beschouwd als er aan de hand van deze resultaten een duidelijk verschil tussen de omgevingen zichtbaar is 
en op basis daarvan een conclusie met advies kan gegeven worden.
\section{\IfLanguageName{dutch}{Opzet van deze bachelorproef}{Structure of this bachelor thesis}}%
\label{sec:opzet-bachelorproef}

% Het is gebruikelijk aan het einde van de inleiding een overzicht te
% geven van de opbouw van de rest van de tekst. Deze sectie bevat al een aanzet
% die je kan aanvullen/aanpassen in functie van je eigen tekst.

De rest van deze bachelorproef is als volgt opgebouwd:

In hoofdstuk~\ref{ch:stand-van-zaken} wordt een overzicht gegeven van de stand van zaken binnen het onderzoeksdomein,
 op basis van een literatuurstudie.

In hoofdstuk~\ref{ch:methodologie} wordt de methodologie toegelicht en worden de gebruikte onderzoekstechnieken besproken om een antwoord te kunnen formuleren 
op de onderzoeksvragen.

In hoofdstuk~\ref{ch:selectie} wordt de lijst van omgevingen afgetoetst aan de opgegegeven requirements om uiteindelijk een geschikte omgeving te selecteren.

In hoofdstuk~\ref{ch:proof-of-concept} wordt de opzet en metingen van de proof-of-concepts toegelicht. Op het einde worden de bekomen resultaten besproken.

In hoofdstuk~\ref{ch:conclusie}, tenslotte, 
wordt de conclusie gegeven en een antwoord geformuleerd op de onderzoeksvragen. 
Daarbij wordt ook een aanzet gegeven voor toekomstig onderzoek binnen dit domein.