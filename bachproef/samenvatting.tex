%%=============================================================================
%% Samenvatting
%%=============================================================================

% TODO: De "abstract" of samenvatting is een kernachtige (~ 1 blz. voor een
% thesis) synthese van het document.
%
% Een goede abstract biedt een kernachtig antwoord op volgende vragen:
%
% 1. Waarover gaat de bachelorproef?
% 2. Waarom heb je er over geschreven?
% 3. Hoe heb je het onderzoek uitgevoerd?
% 4. Wat waren de resultaten? Wat blijkt uit je onderzoek?
% 5. Wat betekenen je resultaten? Wat is de relevantie voor het werkveld?
%
% Daarom bestaat een abstract uit volgende componenten:
%
% - inleiding + kaderen thema
% - probleemstelling
% - (centrale) onderzoeksvraag
% - onderzoeksdoelstelling
% - methodologie
% - resultaten (beperk tot de belangrijkste, relevant voor de onderzoeksvraag)
% - conclusies, aanbevelingen, beperkingen
%
% LET OP! Een samenvatting is GEEN voorwoord!

%%---------- Nederlandse samenvatting -----------------------------------------
%
% TODO: Als je je bachelorproef in het Engels schrijft, moet je eerst een
% Nederlandse samenvatting invoegen. Haal daarvoor onderstaande code uit
% commentaar.
% Wie zijn bachelorproef in het Nederlands schrijft, kan dit negeren, de inhoud
% wordt niet in het document ingevoegd.

\IfLanguageName{english}{%
\selectlanguage{dutch}
\chapter*{Samenvatting}
\lipsum[1-4]
\selectlanguage{english}
}{}

%%---------- Samenvatting -----------------------------------------------------
% De samenvatting in de hoofdtaal van het document

\chapter*{\IfLanguageName{dutch}{Samenvatting}{Abstract}}

Performantie is doorheen de tijd alsmaar belangrijker geworden. 
Een goed presterende applicatie draagt bij aan een betere gebruikerservaring.
Gedurende deze tijd zijn er voortdurend nieuwe JavaScript runtime-omgevingen ontwikkeld met telkens nieuwe verbeteringen om
zo de noden van de alsmaar complexere applicaties te vervullen. 
Ondanks deze nieuwe ontwikkelingen worden weinig ervan effectief in de praktijk gebruikt.
Zo wordt het oudere Node.js vaak nog gekozen als JavaScript runtime-omgeving bij de ontwikkeling van applicaties.
Dit onderzoek heeft als doel inzicht te verschaffen in de performantie van Node.js en één van deze nieuwe omgevingen
om zo een correcte keuze te kunnen maken bij de ontwikkeling van performante applicaties.
De onderzoeksvraag hierbij is of één van deze nieuwe frameworks  
een correcte plaatsvervanger kan zijn voor Node.js bij de ontwikkeling van applicaties binnen bedrijven 
waar performantie centraal staat. Het onderzoek kan dan gebruikt worden als basis 
voor het maken van een doordachte backend keuze.
Om een antwoord te bekomen werd eerst informatie over het onderzoeksdomein verzameld om zo een duidelijk beeld ervan te schetsen.
Hierbij werden in de literatuurstudie Javascript runtime-omgevingen zoals Node.js, Deno en Bun besproken.
Nadien wordt aan de hand van een requirements-analyse één omgeving geselecteerd die het meest geschikt is als potentiële plaatsvervanger voor Node.js.
Hierbij voldeed Bun aan alle requirements en werd er gekozen om deze omgeving te vergelijken met Node.js in de proof-of-concepts.
Aan de hand van het Quick Sort algoritme werd de performantie bij computationele berekeningen gemeten.
Daarnaast werd ook een applicatie opgesteld met de mogelijkheid om data op te halen en in te voegen. 
Hierbij werd de performantie van beide omgevingen gemeten bij het afhandelen van netwerkverzoeken.
Dezelfde applicatie werd ook gebruikt om de installatietijd van de respectievelijke package managers te meten.
Hierbij heeft Bun een snellere installatietijd dan Node.js.
Bij het meten van het Quick Sort algoritme heeft Bun ook een snellere uitvoeringstijd wat het standpunt 
ondersteunt dat Bun beter presteert bij het uitvoeren van computationele berekeningen.
Tot slot heeft Bun een lagere latentie en een betere verwerking van het aantal verzoeken per seconde bij het ophalen en
invoegen van data ten koste van het CPU-gebruik en geheugengebruik.
Dit leidt tot de conclusie dat Bun een optimale keuze kan zijn voor applicaties met complexe algoritmes en calculaties. Echter, 
bij applicaties waarbij I/O-taken worden uitgevoerd, moet telkens een afweging gemaakt worden tussen de snellere verwerking van Bun
en het hogere middelengebruik dat hiermee gepaard gaat.


