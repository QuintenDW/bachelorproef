%%=============================================================================
%% Conclusie
%%=============================================================================

\chapter{Conclusie}%
\label{ch:conclusie}

% TODO: Trek een duidelijke conclusie, in de vorm van een antwoord op de
% onderzoeksvra(a)g(en). Wat was jouw bijdrage aan het onderzoeksdomein en
% hoe biedt dit meerwaarde aan het vakgebied/doelgroep? 
% Reflecteer kritisch over het resultaat. In Engelse teksten wordt deze sectie
% ``Discussion'' genoemd. Had je deze uitkomst verwacht? Zijn er zaken die nog
% niet duidelijk zijn?
% Heeft het onderzoek geleid tot nieuwe vragen die uitnodigen tot verder 
%onderzoek?

In dit onderzoek werd aan de hand van een selectie gekozen om de performantie van Bun te vergelijken met Node.js om zo een antwoord te vinden op de 
onderzoeksvraag of Bun een correcte plaatsvervanger is van Node.js binnen de ontwikkeling van performante JavaScript applicaties.
In deze selectie werden de omgevingen afgetoets aan een lijst van voorafgedefinieerde requirements. Hierbij voldeed enkel Bun aan alle requirements.
Om deze onderzoeksvraag te kunnen beantwoorden werd aan de hand van de literatuurstudie en proof-of-concept een antwoord gezocht op volgende deelvragen:
\begin{itemize}
    \item Wat zijn de onderliggende verschillen tussen Node.js en Bun?
    \item Is Bun performanter dan Node.js bij het uitvoeren van logische berekeningen?
    \item Is Bun performanter dan Node.js bij het afhandelen van netwerk verzoeken?
    \item Wat is het verschil tussen hun respectievelijke package managers?
  \end{itemize}

De deelvraag betreffende de onderliggende verschillen werd beantwoord doormiddel van de literatuurstudie.
Hierbij werd uitgelegd dat Bun een andere JavaScript engine gebruikt dan Node.js om zo onder andere een snellere opstart tijd te bereiken.
Zo gebruikt Bun de JavaScriptCore engine in plaats van de V8 JavaScript engine die Node.js gebruikt.
Deze engine kan geoptimaliseerde machine code produceren in samenwerking met 3 JIT compilers en een Low-Level interpreter.
Daarnaast wordt ook een andere programmeertaal gebruik bij beide omgevingen. Zo wordt C gebruikt bij Node.js, terwijl Bij Bun werd gekozen voor Zig.
Daarnaast heeft Bun veel zaken ingebouwd die bij Node.js extra moeten worden toegevoegd.
Het gaat hierbij over:
\begin{itemize}
    \item Een ingebouwde Test runner.
    \item Een ingebouwde bundler.
    \item Standaard TypeScript ondersteuning.
\end{itemize}
Uit de metingen van de proof-of-concepts kon een antwoord gevonden op de overige 3 deelvragen.
Hierbij werd voor zowel Bun als Node.js telkens 2 proof-of-concepts opgezet. 
Eén van deze proof-of-concept bestond uit een script dat het Quick Sort algoritme uitvoert om zo de performantie van logische berekeningen te meten.
Daarnaast werd voor de performantie bij network verzoeken te meten een applicatie back-end opgesteld waarbij 
een lijst van onderwerpen kon opgehaald worden en een recensie zelf gemaakt kon worden over één van deze onderwerpen.
Bij deze back-end werd het Sequelize ORM gebruikt in samenwerking met zowel een MySQL databank als een PostgreSQL databank.
Uit de resultaten van het Quick Sort algoritme bleek dat Bun performanter is bij het uitvoeren van logische berekeningen.
Zo had Bun een gemiddelde uitvoeringstijd die 2,21 keer sneller was dan Node.js.
Bij het ophalen van data was er ook een groot performantie verschil tussen beiden. Zo had Bun met zowel de MySQL database als de PostgreSQL database 
een beter gemiddelde op vlak van aantal verzoeken per seconde en latentie. Echter bereikte Bun dit door een hoger gebruik van CPU en geheugen middelen.
Bij het invoegen van data scoorde Bun met een MySQL gelijkaardig aan Node.js op vlak van gemiddeld aantal verzoeken en latentie.
Echter gebruikte Bun hierbij wel weer meer middelen. Bij de PostgreSQL database scoorde Bun wel beter op het gemiddelde aantal verzoeken en latentie,
maar gebruikte het weer een hoger aantal middelen.
Uit deze resultaten kan geconcludeerd worden dat Bun sneller netwerk verzoeken kan afhandelen, met de exceptie van POST verzoeken in combinatie met een MySQL databank,
ten koste van een hoger middelengebruik.
Als laatste werd ook de installatietijd onderzocht bij de package managers. Hierbij had Bun een 
gemiddelde installatietijd van 24,9 milliseconden wat 24,5 keer sneller was dan Node.js welke een gemiddelde installatietijd van 610,1 milliseconde had.
Dit toont een significant verschil tussen de respectievelijke package managers waarbij Bun performanter presteert.
Uit deze metingen blijkt Bun een optimale keuze voor applicaties met complexe algoritmes en calculaties door zijn geoptimaliseerde machine code.
Echter moet bij I/O-taken telkens een afweging worden gemaakt tussen de snellere verwerking van Bun en het hogere middelengebruik dat dit teweeg brengt.


