%%=============================================================================
%% Selectie
%%=============================================================================

\chapter{\IfLanguageName{dutch}{Proof-of-concept}{Proof-of-concept}}%
\label{ch:proof-of-concept}

In dit hoofdstuk wordt de proof-of-concept besproken 
waarbij een beoordeling wordt gevormd van de omgevingen op basis van de performantie metingen.
Allereerst wordt de opzet van de proof-of-concepts voor beide omgevingen besproken.
Hierna worden de performantie metingen uitgevoerd voor beide omgevingen waarna de resultaten worden besproken.
Als laatste worden alle resultaten samengevat op zo het te bekijken of Bun een geschikte plaatsvervanger van Node.js kan zijn.

\subsection{Opzet proof-of-concept}
Voor de performantie te meten wordt er per omgeving 2 proof-of-concepts gemaakt. 
De eerste proof-of-concept zal bestaan uit een script dat het Quick Sort algoritme bevat. 
Deze zal dienen om de computationele verwerking van elke omgeving te beoordelen. Hierbij wordt rekening gehouden met volgende metingen:
\begin{itemize}
    \item De gemiddelde uitvoeringstijd.
    \item Het gemiddelde CPU-gebruik.
    \item Het gemiddelde geheugengebruik.
\end{itemize}
Het Quick Sort algoritme zal een meegegeven array sorteren door een spil element te kiezen binnen de array, 
en hierna de array in 2 arrays opsplitsen. De eerste array zal elementen bevatten die kleiner zijn dan het spil element 
en de andere array zal elementen bevatten die groter zijn. 
De 2 arrays worden vervolgens telkens recursief gesorteerd met dezelfde methode om zo een gesorteerde array te bekomen.
In het codevoorbeeld ~\ref{code:quicksort} kan de code voor het Quick Sort algoritme gevonden worden.
Hierbij is er de mogelijkheid de grootte van de te sorteren array mee te geven via de command-line.

\begin{listing}[H]
    \centering
    \begin{minted}[bgcolor=bg,
        fontfamily=tt,
        linenos=true,
        numberblanklines=true,
        numbersep=5pt,
        gobble=0,
        framesep=2mm,
        tabsize=4,
        obeytabs=false,
        breaklines=true,
        mathescape=false
        samepage=false,
        showspaces=false,
        showtabs =false,
        texcl=false]{js}
const quickSort = (array) => {
  if (array.length <= 1) {
    return array;
  }

  let pivot = array[0];
  let smallArray = [];
  let bigArray = [];

  for (let index = 1; index < array.length; index++) {
    if (array[index] < pivot) {
      smallArray.push(array[index]);
    } else {
      bigArray.push(array[index]);
    }
  }

  return [...quickSort(smallArray), pivot, ...quickSort(bigArray)];
};

const args = process.argv.slice(1); // get length of array
let myArray = Array.from({ length: args[1] }, () =>
  Math.floor(Math.random() * 9)
);
quickSort(myArray);
        \end{minted}
        \caption{\label{code:quicksort}Code voorbeeld quicksort}
\end{listing}

Naast de computationele verwerking te meten, wordt ook performantie bij I/O-taken gemeten.
Dit wordt gedaan door een proof-of-concept op te stellen die een applicatie back-end voorstelt in de respectievelijke omgeving.
Hierbij kan een gebruiker een recensie creëren over een bepaald onderwerp. 
Binnen de proof-of-concepts zal hierbij getest worden met zowel een Postgres databank als een MySQL databank.
Hierbij wordt rekening gehouden met volgende metingen:
\begin{itemize}
    \item De gemiddelde uitvoeringstijd.
    \item Het gemiddelde CPU-gebruik.
    \item Het gemiddelde geheugengebruik.
    \item De responstijd.
    \item Het aantal gelijktijdige connecties.
    \item Het aantal verzoeken.
\end{itemize}

