%%=============================================================================
%% Voorwoord
%%=============================================================================

\chapter*{\IfLanguageName{dutch}{Woord vooraf}{Preface}}%
\label{ch:voorwoord}

%% TODO:
%% Het voorwoord is het enige deel van de bachelorproef waar je vanuit je
%% eigen standpunt (``ik-vorm'') mag schrijven. Je kan hier bv. motiveren
%% waarom jij het onderwerp wil bespreken.
%% Vergeet ook niet te bedanken wie je geholpen/gesteund/... heeft

Met trots presenteer ik mijn bachelorproef "De beste JavaScript runtime-omgeving voor performante applicaties: een vergelijkende studie", 
waarin ik, in het kader voor het voltooien van mijn opleiding Toegepaste Informatica, Node.js vergelijk met het opkomende Bun op vlak van performantie.
Tijdens mijn studiejaren ben ik altijd geïnteresseerd geweest in JavaScript runtime-omgevingen zoals onder andere Node.js en Bun en
hoe deze tegen over elkaar staan. Toen ik een onderwerp moest kiezen kwam dit direct in me op aangezien het mij interessant leek om me dieper te verdiepen 
in Bun. Het schrijven van deze bachelorproef heeft me dan ook uitgedaagd om mijn kennis en vaardigheden te verbreden.
\vspace{5mm}

Ik wil mijn oprechte dank betuigen aan iedereen die heeft bijgedragen aan deze bachelorproef, 
in het bijzonder mijn promotor Mevr M. Van Audenrode en mijn co-promotor Mr. D. Van Der Elst voor hun
waardevolle begeleiding en feedback gedurende het hele proces. 
Zij stonden altijd klaar voor vragen en gaven telkens constructieve en gerichte feedback waarvoor ik hun zeer dankbaar ben.
\vspace{5mm}

Tot slot wil ik mijn ouders bedanken voor de kansen die zij mij hebben gegeven. 
Dankzij hun dagelijkse inzet kan ik mij telkens concentreren op mijn studies.


