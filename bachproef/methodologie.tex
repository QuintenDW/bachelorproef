%%=============================================================================
%% Methodologie
%%=============================================================================

\chapter{\IfLanguageName{dutch}{Methodologie}{Methodology}}%
\label{ch:methodologie}

%% TODO: In dit hoofstuk geef je een korte toelichting over hoe je te werk bent
%% gegaan. Verdeel je onderzoek in grote fasen, en licht in elke fase toe wat
%% de doelstelling was, welke deliverables daar uit gekomen zijn, en welke
%% onderzoeksmethoden je daarbij toegepast hebt. Verantwoord waarom je
%% op deze manier te werk gegaan bent.
%% 
%% Voorbeelden van zulke fasen zijn: literatuurstudie, opstellen van een
%% requirements-analyse, opstellen long-list (bij vergelijkende studie),
%% selectie van geschikte tools (bij vergelijkende studie, "short-list"),
%% opzetten testopstelling/PoC, uitvoeren testen en verzamelen
%% van resultaten, analyse van resultaten, ...
%%
%% !!!!! LET OP !!!!!
%%
%% Het is uitdrukkelijk NIET de bedoeling dat je het grootste deel van de corpus
%% van je bachelorproef in dit hoofstuk verwerkt! Dit hoofdstuk is eerder een
%% kort overzicht van je plan van aanpak.
%%
%% Maak voor elke fase (behalve het literatuuronderzoek) een NIEUW HOOFDSTUK aan
%% en geef het een gepaste titel.

Het onderzoek bevat 4 fasen. 
De eerste fase bestaat uit het verzamelen van informatie over het onderzoeksdomein om zo een duidelijk beeld ervan te schetsen.
Dit wordt gedaan aan de hand van een literatuurstudie die te vinden is in hoofdstuk \ref{ch:stand-van-zaken}. 
Hierbij wordt eerst een algemene beschrijving over Javascript runtime-omgevingen besproken waarna 
er dieper wordt ingegaan op Node.js alsook zijn opvolger Deno en de nieuwe runtime Bun. 
Specifiek worden hun respectievelijke package managers, javascript engines, bundlers en extra functionaliteiten besproken.
Nadat er een duidelijker beeld is over het onderzoeksdomein wordt in de tweede fase, aan de hand van een requirements-analyse, 
één omgeving geselecteerd die het meest geschikt is als potentiële plaatsvervanger voor Node.js binnen de context van performante applicaties.
De requirements worden hier onderverdeeld via de MoSCoW-techniek in de categorieën “must-have”, “should-have”, “could-have” en “won't-have”. 
De alternatieven worden afgetoetst aan de requirements om zo tot één geschikte omgeving te komen die wordt vergeleken met Node.js.
In de derde fase worden zelfgemaakte proof-of-concepts opgesteld en getest (zie hoofdstuk \ref{label}) voor beide omgevingen. 
Hierbij wordt een applicatie ontwikkeld waarbij een gebruiker
een recensie kan creëren over een bepaald onderwerp,
alsook een script dat een algoritme bevat. Hierdoor kunnen zowel metingen gedaan worden op computationele taken als I/O-taken (Input/Output).
De metingen worden uitgevoerd door 2 benchmark tools, Hyperfine en Bombardier, om volgende zaken te meten:
\begin{itemize}
    \item De responstijd op HTTP-verzoeken
    \item De uitvoeringstijd bij de applicatie.
    \item De verwerkingstijd bij het uitvoeren van de berekeningen van het algoritme.
    \item Het geheugengebruik bij beiden.
    \item De installatietijd voor de respectievelijke package managers.
    \item Het CPU-gebruik bij beiden.
\end{itemize}
Daarnaast zal ook gekeken worden naar de extra functionaliteiten en hoe deze de complexiteit beïnvloeden.
Aan de hand van deze resultaten komen we tot de laatste fase namelijk de conclusie. 
Hierbij worden de data geanalyseerd om zo een antwoord te formuleren op de onderzoeksvragen.
Hieruit wordt afgeleid of het gekozen framework een geschikte plaatsvervange kan zijn voor Node.js binnen 
de ontwikkeling van webapplicaties waar performantie centraal staat.
